% !TEX program = xelatex
\documentclass[
  10pt,
  twoside,
  openany,
  b5paper, % 以上均为 ctexbook 提供的文类选项
  colorscheme = basic, % 请根据需要选择或定制配色方案
]{qyxf-book}

\usepackage{graphicx}
\usepackage{epstopdf}
\usepackage[journal=jacs]{chemstyle} 
\usepackage{subcaption}
\usepackage{ccicons}
\usepackage{draftwatermark}
\usepackage{fontawesome5}
\SetWatermarkText{振宇考研}
\SetWatermarkLightness{0.92}
\SetWatermarkScale{0.9}

\title{有机化学六脉神剑}
\subtitle{Six Veins Excalibur of Organic Chemistry }  % 可选
\author{韩教练,喵小决}
\date{2023 年 3 月 1 日}
\typo{喵小决,于小花}  % 排版人员信息,选填

% 定制元信息
\org{\Large\textit{振宇考研}\\\textsc{CLAUSIUS ZHEN YU KAO YAN}}
\footorg{\textsc{ZHEN YU KAO YAN}}
\cover{\includegraphics[width=.6\textwidth]{logo.pdf}}
\license{}  % 清空许可证信息

% 调整封面标题大小
\renewcommand{\titlefont}{\Huge\bfseries}
\renewcommand{\subtitlefont}{\LARGE\itshape}

\begin{document}

\maketitle

\chapter*{前言}

受振宇考研韩教练的邀请,为这本《有机化学六脉神剑》习题书做有机结构的绘制与全书的排版工作。这是一个相当巨大的工程,在实际排版过程中,也经历了些许难题和挑战,感谢振宇考研的韩教练对本人的信任与支持。

本书大胆的使用了\LaTeX 与Chemdraw相结合的方式来编写有机化学资料,使用了\href{https://github.com/qyxf/qyxf-book}{钱院学府}的 \LaTeX 模板,在此特别鸣谢。本书创新的使用了chemstyle宏包来在\LaTeX 中插入有机结构式,选择了该宏包中的jacs主题作为全书的有机结构风格。下面是使用该宏包在有机化合物绘制的效果:

\begin{scheme}[ht]
	\includegraphics{eg/eg1.eps}
\end{scheme}



本作品采用\href{https://
	creativecommons.org/licenses/
	by-nc-nd/4.0/}{ BY-
	NC-ND 4.0 协议}进行许可。使用者可以在给出作者署名及资料来源的前提下对本作品进行转载,但不得对本作品进行修改,亦不得基于本作品进行二次创作,不得将本作品运用于商业用途。

由于本书的大部分内容都由本人绘制并排版,但笔误、错漏等在所难免,如您在参考的过程中发现有任何错误之处,欢迎您通过下面的方式联系我们,帮助我们改进这份习题:
\begin{itemize}
	\item \faGithub ~~ GitHub平台论坛:\url{https://github.com/kimariyb/organic-chemistry/issues}
	\item \faEnvelopeOpen ~~ 韩教练邮箱:\texttt{1377692745@qq.com}
	\item \faQq ~~ 
	本人QQ:~~\textbf{喵小决}~~2420707848
\end{itemize}

\begin{flushright}
	喵小决\\
	2023 年 3 月 1 日
\end{flushright}

\chapter*{编者序}



\begin{flushright}
	韩教练\\
	2023 年 3 月 1 日
\end{flushright}

\cleardoublepage


\tableofcontents

\chapter{烷烃、脂环烃、立体化学}

\section{选择题}

\exercise{1} 「郑州大学2022」 (1S,\ 2R)-1-苯基-2-甲氨基-1-丙醇的Fischer投影式为
\begin{scheme}[ht]
	\includegraphics{chapter1/01/001.eps}
\end{scheme}

\section{完成反应题}
\section{合成题}
\section{机理题}
\section{实验题}
\section{综合题}

\chapter{烯烃、炔烃、共轭烯烃}

\section{选择题}
\section{完成反应题}
\section{合成题}
\section{机理题}
\section{实验题}
\section{综合题}

\chapter{卤代烃}

\section{选择题}
\section{完成反应题}
\section{合成题}
\section{机理题}
\section{实验题}
\section{综合题}

\chapter{醇、醚、脂肪胺}

\section{选择题}
\section{完成反应题}
\section{合成题}
\section{机理题}
\section{实验题}
\section{综合题}

\chapter{芳香烃、芳香胺、杂环}

\section{选择题}
\section{完成反应题}
\section{合成题}
\section{机理题}
\section{实验题}
\section{综合题}

\chapter{醛、酮、羧酸}

\section{选择题}
\section{完成反应题}
\section{合成题}
\section{机理题}
\section{实验题}
\section{综合题}

\chapter{羧酸衍生物}

\section{选择题}
\section{完成反应题}
\section{合成题}
\section{机理题}
\section{实验题}
\section{综合题}

\chapter{周环反应、糖和氨基酸}

\section{选择题}
\section{完成反应题}
\section{合成题}
\section{机理题}
\section{实验题}
\section{综合题}


\end{document}